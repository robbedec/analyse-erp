%%====================================================
%% Vragen
%%====================================================

\chapter{Appendices}
\label{ch:vragen}
\section{Vragen interview}
\begin{enumerate}
    \item Wat is de link tussen het geïnterviewde bedrijf en ERP
    
    \item Wanneer kiest een klant voor een ERP systeem
    
    \item Vergemakkelijkt het requirements management process als je werkt met off-the-shelf ERP software
     
    \item Is het zinvol om requirements management process op te stellen bij ERP
    
    \begin{enumerate}
        \item In welke mate wordt het systeem nog gepersonaliseerd (+ veel voorkomende modules die gepersonaliseerd worden)
        
        \item Op welke manier verzamelen jullie de bedrijfsprocessen (BPMN...)
    \end{enumerate}

    \item Kan de implementatie van zo'n systeem agile verlopen als er geen volledig overzicht is van de requirements
    
    \begin{enumerate}
        \item Hoe verloopt de integratie in het bedrijf
    \end{enumerate}
    
    \item Waarom verkiezen bedrijven een SAP implementatie boven een kleiner op maat gemaakt ERP systeem
    
    \begin{enumerate}
        \item Kan een kleiner systeem voordeliger zijn
    \end{enumerate}
    
    \item Zijn er ook nadelen verbonden aan het werken aan / met ERP
    
    \begin{enumerate}
        \item Aan welke zaken werkt het bedrijf actief om efficiënter te werken
        
        \item Hoe groot is de learning curve voor iemand die voor het eerst in contact komt met zo'n pakket
        
        \item Wat met klanten die achteraf nog veranderingen willen
    \end{enumerate}

    \item Hoe lang duurt het om een volledig systeem te leveren
    
    \item Zijn er werknemers die vrezen dat hun job geautomatiseerd wordt
    
    \item Hoe voordelig is een ERP-implementatie voor een klein bedrijf
    
    \begin{enumerate}
        \item M.a.w. op welk soort bedrijf ligt de focus (klein, kmo, grote onderneming...)
        
        \item Waar merk je duidelijk de return of investment
    \end{enumerate}
\end{enumerate}