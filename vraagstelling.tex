%%====================================================
%% Vragen
%%====================================================

\chapter{Vragen}
\label{ch:vraagstelling}

De vragen zijn opgesteld aan de hand van de literatuurstudie. Er wordt gecontroleerd of de opgestelde hypothese correct is bij maatwerk en parametrisatie. Beide interviews starten met het schetsen van de bedrijfssituatie, link tussen dagelijkse taken en ERP, welke voor- en nadelen zijn er bij het werken aan / met ERP en soortgelijke vragen. Omdat we samen zitten met een bedrijf dat enkel met SAP werkt en een andere die een eigen ERP systeem heeft ontwikkeld vragen we ook waarom zoveel bedrijven kiezen voor SAP en of een kleiner systeem ook voordeliger kan zijn en in welke situaties. 

De essentie van het onderzoek gaat over het verzamelen van de requirements en wordt ook gezien als 1 van de grootste stappen van het proces. Uit de literatuurstudie blijkt dat in 1 iteratie alle requirements verzamelen praktisch onmogelijk en men zal dus een agile systeem moeten toepassen \autocite{Vollmer2016}. We vragen ons af hoe zo'n implementatie agile kan verlopen en wat voor gevolgen dat heeft voor de integratie bij de klant. Over het requirements management proces zelf willen we graag weten hoe zinvol dit is (in het geval dat het systeem veel gepersonaliseerde modules bevat), op welke manier ze dan verzameld worden en de impact van requirements op de oplevering na sprint X.

Op het einde van het interview hebben we ook nog wat randinformatie verzameld met als doel het concept en de implementatie beter te begrijpen. Voorbeelden hiervan zijn: bedrijven waarop de focus ligt, de grote voordelen van hun product, de impact op de return on investment, de learning curve voor de klant en hoe ze om gaan met wijzigingen in het systeem.