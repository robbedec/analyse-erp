%%====================================================
%% Vragen
%%====================================================

\chapter{Vragen}
\label{ch:vraagstelling}

De vragen zijn opgesteld aan de hand van de literatuurstudie. Er wordt dus gecontroleerd als de opgestelde hypothese: `Maatwerk en parametrisatie heeft een grote impact op het requirements management proces en is nodig om een ERP pakket te kiezen.` correct is in de praktijk. De beide interviews starten met het schetsen van het bedrijf hun situatie, wat is de link tussen dagelijkse taken en ERP, welke voor- en nadelen zijn er bij het werken aan / met ERP en soortgelijke vragen. Omdat we samen zitten met een bedrijf dat enkel met SAP werkt en een andere die een eigen ERP systeem heeft ontwikkeld vragen we ook waarom zoveel bedrijven kiezen voor SAP en als een kleiner systeem ook voordeliger kan zijn en in welke situatie. 

De essentie van het onderzoek gaat over het verzamelen van de requirements en wordt ook gezien als 1 van de grootste stappen van het proces. Uit de literatuurstudie blijkt dat in 1 iteratie alle requirements verzamelen praktisch onmogelijk is dus men zal een agile systeem moeten toepassen \autocite{Vollmer2016}. We vragen ons dus af hoe zo'n implementatie agile kan verlopen en wat voor gevolgen dat heeft voor de integratie bij de klant. Over het requirements management proces zelf willen we graag weten hoe zinvol dit is (in het geval dat het systeem veel gepersonaliseerde modules bevat), op welke manier ze dan verzameld worden en de impact op een 1e oplevering na sprint X.

Op het einde van het interview verzamelen we wat randinformatie die minder bruikbaar zijn om een antwoord op de hypothese te vinden maar die wel handig zijn voor ons om het concept en de implementatie beter te begrijpen. Deze gaan over de grote voordelen dat hun product aanbiedt en waar je de return of investment het snelst zal opmerken en op welk soort bedrijven hun focus vooral ligt. Hoe men in de toekomst nog veranderingen in het systeem wilt, waaraan het bedrijf actief werkt om een efficiëntere dienstverlening te bekomen en hoe gebruikers de learning curve van zo'n pakket kunnen overwinnen.