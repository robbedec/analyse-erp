%%====================================================
%% Conclusie
%%====================================================

\chapter{Conclusie}
\label{ch:conclusie}

Uit dit onderzoek kunnen we concluderen dat het requirements management process van groot belang is bij het ontwikkelen van erp-systemen. Het verloop van dit process hangt af van de bedrijfscontext en kan dus zeker variëren. Wel valt op dat men het opstellen van de requirements niet lauter aan de klant overlaat. Soms worden er zelfs bedrijven gebruikt die zich specialiseren in het ontwikkelen van deze requirements omdat het zeer belangrijk is dat dit correct gebeurd. Het spreekt voor zich dat de klant niet tevreden zal zijn met een erp-systeem dat ontwikkeld is op basis van verkeerd opgestelde requirements. Om tijd en geld te besparen is het correcte verloop van het requirements management process dus van cruciaal belang. 

Agile werken bij een ERP-implementatie blijft mogelijk, maar in de praktijk gebeurt dit niet altijd. De keuze om agile te werken zal hier afhangen van de soort klant. Bij 'nieuwe klanten' zal er zelden iteratief gewerkt worden, terwijl er bij bestaande klanten waarbij het project vaak groter zal zijn, vaker gekozen wordt om via agile in sprints te werken.