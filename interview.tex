%%====================================================
%% Interview
%%====================================================

\chapter{Analyse Interview}
\label{ch:analyse_interview}

\section{Expertum}

Om kort de situatie van het bedrijf te schetsen, is deze beknopte inleiding voorzien. Expertum is een grotendeels een leverancier van SAP systemen. Je kan bij hen terecht zonder enige voorkennis of om een huidig systeem over te nemen. Hun 2e grote activiteit is het ontwikkelen van applicaties die samenwerken met een SAP systeem (al dan niet in opdracht van een bedrijf) waarbij closed source wordt toegepast. Het is belangrijk om op te merken is dat aanpassingen aan SAP systemen verplicht zijn om open source te zijn in de zin dat de klant de volledige code krijgt, voor services die niet helemaal verwerkt zijn in het systeem mogen ze de code voor zichzelf houden en werken met een SaaS model. Verder leveren ze nog support en licenties.

Uit het interview blijkt dat het verzamelen van de requirements 1 van de belangrijkste stappen van het proces is en kan bepalen als je project zal slagen of niet, dit komt sterk overeen met de theorie waarbij dat proces telkens de 1e stap zal zijn. In de praktijk is het niet de klant die zijn eigen requirements zal verzamelen (er wordt ons de vraag gesteld van welke klant weet nu juist wat ze exact willen) maar ze zullen een Request for Proposal op de markt gooien. Er zijn gespecialiseerde bedrijven die dat zeer specifiek voor je bedrijf zullen maken. De volgende stap is een Request for Information opstellen, dit is zeker nodig als je een ERP pakket wilt maar niet weet wat er allemaal beschikbaar is. Hun link met ERP is dat Expertum zal antwoorden op een Request for Information, advies geven en de implementatie maken. Omdat SAP zo'n grote verzameling is van processen zijn er standaard schema's voorzien van alle functionaliteiten die in de default library zitten. Deze zijn gemaakt volgens de BPMN standaard maar enkel beschikbaar voor bedrijven met een actieve licentie. 

De bedoeling van SAP is dat er een goede basis beschikbaar is voor iedereen. Dit zijn vaak triviale zaken zoals een verkoop-proces die in vrij wel elke situatie op dezelfde manier verloopt, de flow van zo'n proces kan je volgen in de meegeleverde BPMN schema's die eerder vermeld zijn. Deze basis kan je uitbreiden met verschillende modules die ook een standaardimplementatie hebben en simpel gezegd moet linken aan het systeem en hun schakelaar activeren. Dit is een beslissing die SAP heeft gemaakt om van de negatieve connotatie van het verleden af te raken. Toen was er geen enkele standaard voorzien en alle leveranciers begonnen te programmeren maar er was geen enkele garantie over de kwaliteit van de code of dat je systeem overdraagbaar zou zijn mocht de leverancier failliet gaan. Op deze manier kan je je verwachten aan performante code die in theorie binnen 20 jaar nog altijd ondersteund wordt. Zoals in de literatuurstudie wordt ook hier een extra push gegeven door de leverancier om toch te proberen binnen deze (uitgebreide) basis te blijven, extra modules te activeren en waar nodig en in speciale gevallen zelf een module programmeren. Trouwens kan elke leverancier een bijdrage leveren aan de standaard als het een veel gevraagde functionaliteit is, deze wordt dan geëvalueerd door een panel bij SAP en ook door programmeurs daar gemaakt. De officiële benaming van een bedrijf zoals Expertum is een Value Added Reseller en praktisch iedereen moet via deze weg als ze een SAP implementatie willen, enkel de hele grote klanten kunnen rechtstreeks aan een licentie raken.

Met de laatste reeks vragen vergelijken we bevindingen over het agile implementeren. Vroeger ging men feature per feature gaan programmeren, beter bekend als de waterval-methode. Een eerste stap richting agile werken was het opzetten van een blueprint fase waarin een backlog wordt opgesteld, deze bevatte meer details dan een moderne backlog omdat deze niet meer opgesplitst werd in user stories. Het grootste deel van de implementatie verloopt niet iteratief maar er wordt overgeschakeld op het moment dat de klant input geeft, wat dus niet nuttig is als je de standaard modules van SAP aan elkaar moet linken.


\section{Zero-friction}

Zero-friction maakt ERP pakketten voor de warmte markt. Zo zou deze markt heel specifieke behoeften hebben waarbij de software die zij ontwikkelen heel nauw op aan moet sluiten. Ze gaan parameterisatie toepassen maar het maatwerk doen ze voorlopig niet. Ze laten de klant niet toe op aanpassingen te doen op de flow van de software. Naar de toekomst toe zouden ze dit eventueel wel toestaan. De reden hiervoor is om de upgrade kost zo laag mogelijk te houden. Configuraties aan bijvoorbeeld factuurtemplates kunnen uiteraard wel worden uitgevoerd.