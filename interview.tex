%%====================================================
%% Interview
%%====================================================

\chapter{Interview}
\label{ch:interview}

\section{Expertum}

Om kort de situatie van het bedrijf te schetsen, is deze beknopte inleiding voorzien. Expertum is een grotendeels een leverancier van SAP systemen. Je kan bij hen terecht zonder enige voorkennis of om een huidig systeem over te nemen. Hun 2e grote activiteit is het ontwikkelen van applicaties die samenwerken met een SAP systeem (al dan niet in opdracht van een bedrijf) waarbij closed source wordt toegepast. Het is belangrijk om op te merken is dat aanpassingen aan SAP systemen verplicht zijn om open source te zijn in de zin dat de klant de volledige code krijgt, voor services die niet helemaal verwerkt zijn in het systeem mogen ze de code voor zichzelf houden en werken met een SaaS model.

Uit het interview blijkt dat het verzamelen van de requirements 1 van de belangrijkste stappen van het proces is en kan bepalen als je project zal slagen of niet, dit komt sterk overeen met de theorie waarbij dat proces telkens de 1e stap zal zijn. In de praktijk is het niet de klant die zijn eigen requirements zal verzamelen (er wordt ons de vraag gesteld van welke klant weet nu juist wat ze exact willen) maar ze zullen een Request for Proposal op de markt gooien. Er zijn gespecialiseerde bedrijven die dat zeer specifiek voor je bedrijf zullen maken. De volgende stap is een Request for Information opstellen, dit is zeker nodig als je een ERP pakket wilt maar niet weet wat er allemaal beschikbaar is. Hun link met ERP is dat Expertum die zal antwoorden op een Request for Information, advies geven en de implementatie maken. Omdat SAP zo'n grote verzameling is van processen zijn er standaard schema's voorzien van alle functionaliteiten die in de default library zitten. Deze zijn gemaakt volgens de BPMN standaard maar enkel beschikbaar voor bedrijven met een actieve licentie. 

\section{Zero-friction}