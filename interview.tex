%%====================================================
%% Interview
%%====================================================

\chapter{Analyse Interview}
\label{ch:analyse_interview}

\section{Expertum}

Om kort de situatie van het bedrijf te schetsen, is deze beknopte inleiding voorzien. Expertum is een grotendeels een leverancier van SAP systemen. Je kan bij hen terecht zonder enige voorkennis of om een huidig systeem over te nemen. Hun 2e grote activiteit is het ontwikkelen van applicaties die samenwerken met een SAP systeem (al dan niet in opdracht van een bedrijf) waarbij closed source wordt toegepast. Het is belangrijk om op te merken is dat aanpassingen aan SAP systemen verplicht zijn om open source te zijn in de zin dat de klant de volledige code krijgt, voor services die niet helemaal verwerkt zijn in het systeem mogen ze de code voor zichzelf houden en werken met een SaaS model. Verder leveren ze nog support en licenties.

Uit het interview blijkt dat het verzamelen van de requirements één van de belangrijkste stappen van het proces is en kan bepalen als je project zal slagen of niet, dit komt sterk overeen met de theorie waarbij dat proces telkens de eerste stap zal zijn. In de praktijk is het niet de klant die zijn eigen requirements zal verzamelen (er wordt ons de vraag gesteld van welke klant weet nu juist wat ze exact willen) maar ze zullen een Request for Proposal op de markt gooien. Er zijn gespecialiseerde bedrijven die dat zeer specifiek voor je bedrijf zullen maken. De volgende stap is een Request for Information opstellen, dit is zeker nodig als je een ERP pakket wilt maar niet weet wat er allemaal beschikbaar is. Hun link met ERP is dat Expertum zal antwoorden op een Request for Information, advies geven en de implementatie maken. Omdat SAP zo'n grote verzameling is van processen zijn er standaard schema's voorzien van alle functionaliteiten die in de default library zitten. Deze zijn gemaakt volgens de BPMN standaard maar enkel beschikbaar voor bedrijven met een actieve licentie. 

De bedoeling van SAP is dat er een goede basis beschikbaar is voor iedereen. Dit zijn vaak triviale zaken zoals een verkoop-proces die in vrij wel elke situatie op dezelfde manier verloopt, de flow van zo'n proces kan je volgen in de meegeleverde BPMN schema's die eerder vermeld zijn. Deze basis kan je uitbreiden met verschillende modules die ook een standaardimplementatie hebben en simpel gezegd moet linken aan het systeem en hun schakelaar activeren. Dit is een beslissing die SAP heeft gemaakt om van de negatieve connotatie van het verleden af te raken. Toen was er geen enkele standaard voorzien en alle leveranciers begonnen te programmeren maar er was geen enkele garantie over de kwaliteit van de code of dat je systeem overdraagbaar zou zijn mocht de leverancier failliet gaan. Op deze manier kan je je verwachten aan performante code die in theorie binnen 20 jaar nog altijd ondersteund wordt. Zoals in de literatuurstudie wordt ook hier een extra push gegeven door de leverancier om toch te proberen binnen deze (uitgebreide) basis te blijven, extra modules te activeren en waar nodig en in speciale gevallen zelf een module programmeren. Trouwens kan elke leverancier een bijdrage leveren aan de standaard als het een veel gevraagde functionaliteit is, deze wordt dan geëvalueerd door een panel bij SAP en ook door programmeurs daar gemaakt. De officiële benaming van een bedrijf zoals Expertum is een Value Added Reseller en praktisch iedereen moet via deze weg als ze een SAP implementatie willen, enkel de hele grote klanten kunnen rechtstreeks aan een licentie raken.

Met de laatste reeks vragen vergelijken we bevindingen over het agile implementeren. Vroeger ging men feature per feature gaan programmeren, beter bekend als de waterval-methode. Een eerste stap richting agile werken was het opzetten van een blueprint fase waarin een backlog wordt opgesteld, deze bevatte meer details dan een moderne backlog omdat deze niet meer opgesplitst werd in user stories. Het grootste deel van de implementatie verloopt niet iteratief maar er wordt overgeschakeld op het moment dat de klant input geeft, wat dus niet nuttig is als je de standaard modules van SAP aan elkaar moet linken.


\section{Zero-friction}

Tegenover Expertum, een bedrijf dat het grote SAP gaat implementeren en configureren bij bedrijven hebben we ook Zero Friction, dit is een bedrijf die een zeer specifiek ERP-pakket ontwikkelt voor de warmte markt. Gaande van het tracken en analyseren van warmteleveringen tot facturatie. Vaak zijn de bekende ERP systemen heel erg groot en heel erg duur, iets dat de warmte markt vaak niet nodig heeft en ook niet kan betalen. Een goed voorbeeld is het bedrijf waar de oprichters van Zero Friction oorspronkelijk vandaan kwamen. Hier gebruikt men Microsoft Dynamics, dit kost het bedrijf erg veel geld en is niet specifiek genoeg voor wat toch een erg niche markt is, hierdoor is erg veel maatwerk nodig. Daarom werd beslist om Zero Friction op te richten, een schaalbaar en goedkoop alternatief voor de warmte markt.

Tijdens het interview werd dit concreet voorgesteld als "Focus en Prijs". Waarbij grote ERP pakketten vaak veel configuratie en maatwerk nodig hebben heeft het pakket van Zero Friction dit niet nodig, ze behouden wel de mogelijkheid tot eventuele parametrisatie, bijvoorbeeld templates, product items, billing-items en meer. Het is dus de bedoeling dat de flow van het programma niet kan aangepast worden door de klanten. Een derde voordeel is dat het veel gebruiksvriendelijker is om in te stappen dan bijvoorbeeld SAP, hierdoor is de trainingsinvestering eerder beperkt wat verder de prijs drukt.

Er zijn ook enkele nadelen, uit het interview blijkt dat het de bedoeling is om dit pakket te combineren met andere ERP pakketten, bijvoorbeeld een boekhoudpakket. Terwijl dit als voordeel aanzien kan worden omdat het de prijs drukt brengt dit ook wel het nadeel met zich mee dat interfaces nodig zijn om verschillende pakketten aan elkaar te koppelen.

Om dit pakket goed op te bouwen is er een goed requirements process nodig, hier staat voor hen de klant centraal, dit betekend dat ze luisteren naar de algemene markt, maar ook individueel naar elke klant. Dit wordt aangepakt aan de hand van een "User Voice", dit zijn feature voorstellen waarop klanten kunnen stemmen en zeggen welke prioriteit moeten krijgen. Zo wordt dus een pakket ontwikkelt dat niet verandert, waar telkens nieuwe features bijkomen. Deze worden meestal voorgesteld door functionele designs die afgetoetst zijn met de verschillende klanten. 

Dit alles gebeurt eveneens op een Agile manier. Al zijn hier wel twee grote verschillen, namelijke nieuwe klanten en al klanten die al omzet hebben. Bij nieuwe klanten is dit erg simpel, hier kan gewoon een nieuw systeem worden opgezet en kunnen alle cliënten van de klant toegevoegd, waarbij een cliënt een klant is van de klant, dit gaat dan ook zeer snel. Indien we naar bestaande klanten kijken is dit een ander verhaal, er is veel meer werk aan. Dit toont zich in de vorm van voornamelijk data-migratie, wat meestal in cliënt-groepen gebeurt, maar ook aanpassingen aan facturen en configuraties.

Net als de meeste ERP pakketten biedt Zero Friction ook SaaS aan, dit zorgt er wel voor dat data niet verloren gaat indien ze failliet zou gaan of de klant wil veranderen van provider. Wel betekent dit dat de Code hun IP (intellectual property) is, dit houdt in dat indien Zero Friction failliet zou gaan men niet zomaar met dezelfde code kan verder doen en men dus opnieuw zou moeten beginnen met een ander ERP pakket.
